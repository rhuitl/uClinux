\subsection{010: routing above tunnel layer}

%  3.1   Multiple Tunnels to Same Subnet
%   
%   Suppose  West is a FreeS/WAN box that is connected to two different wild-side ISPs. It has
%   two  different  wild-side  addresses.  It  would be very nice to set up two connections to
%   Sunrise-net.  This  would  allow load-sharing, and it would also improve reliability if we
%   could arrange failover from one connection to the other.
%   Unfortunately,  if  you  try  to  set  up multiple connections to the same subnet (Sunrise
%   subnet)  using  FreeS/WAN  version  1,  the result is fratricide. The second connection is
%   treated as a replacement for the first.
%   You  can  work  around this by not making subnet connections. Instead, as suggested by Joe
%   Patteson,   make   host-to-host   connections   between   East  and  West,  and  then  run
%   GRE-encapsulated  traffic  over  these  connections,  as discussed in section 2.2. You can
%   perform load-balancing and failover with respect to the GRE virtual devices.
%   If East has N ISPs and West has M ISPs, you could conceivably want M�N IPsec connections.
%   Note the following imperfect parallelism:
%     * IPsec  tunnel  mode  is  sometimes  described (imperfectly) as follows: Set up an IPIP
%       tunnel, and then apply transport-mode encryption to the encapsulated traffic.
%     * In  this  case,  we encapsulate using GRE, and then apply transport-mode encryption to
%       the encapsulated traffic.
% 
%    On  the  other  hand,  it is not 100% correct to view tunnel mode as IPIP plus encryption,
%  because  real  tunnel  mode  allows allows the system to verify that the correct subnet is
%   being  carried  over  the  tunnel; that is, the inner headers can be checked. In contrast,
%   using  IPIP+encryption  or GRE+encryption, the inner headers are just data in some obscure
%   format that cannot be checked by the IPsec system per se.
%   On  the  third  hand,  since  FreeS/WAN relies on you, the user, to implement the security
%   policy  anyway,  using the firewalling system, you can perfectly well check the packets as
%   they enter the GRE device. So all the necessary functionality can be achieved.
%   Also  note that the GRE links, as the name implies, were designed with routers in mind, so
%   their up/down state is meaningful, and their metric is meaningful to routing daemons.
%   It  sure  would  be nice if the next-generation IPsec system could do this itself, without
%   requiring  us  to  resort  to  GRE.  We  need  multiple  tunnels  to the same subnet, with
%   meaningful  up/down  state  and  meaningful  metrics. And we need the ipsec device to play
%   nicely with routing daemons.
   
  

\subsubsection{010: Definition of requirement }

%   http://www.quintillion.com/fdis/moat/ipsec+routing/
%especially section 3 therein, i.e.
%   http://www.quintillion.com/fdis/moat/ipsec+routing/#sec-routing-above
%
%That was written over a year ago.  The big ideas haven't changed, but a few 
%details need polishing.
%  *) In particular, there are several places where it speaks of a "device" 
%going down whereas is should say "link" going down.
%
%  *) Also the rationale is given in terms of failover and load-sharing, and 
%we now know that mobility is a third important motivator for doing this.
%
%  *) Also the terminology
%         a) routing above the tunnel
%         b) routing below the tunnel
%has been deemed unnecessarily confusing to Muggles.  It would be better to 
%speak of
%         a) routing the inner headers (plain text)
%         b routing the outer headers (crypto text)
%but the intended meanings are unchanged.
%
%===========
%
%Tangentially relevant remark: Routing above the tunnel is one of the big 
%motivations for having a virtual device.  A detailed look at how this might 
%work, as worked out at OLS, was posted to the list at 02:15 PM 7/30/01 -0400.

see JSD documents.

